\documentclass[11pt]{article}
\usepackage[twoside,inner=2.5cm,outer=1.5cm,paper=a5paper]{geometry}
\usepackage{fixltx2e} % LaTeX patches, \textsubscript
\usepackage{cmap} % fix search and cut-and-paste in Acrobat
\usepackage{ifthen}
\usepackage[T1]{fontenc}
\usepackage[utf8]{inputenc}
\usepackage{graphicx}
\usepackage{longtable,ltcaption,array}
\setlength{\extrarowheight}{2pt}
\newlength{\DUtablewidth} % internal use in tables

\usepackage{mathptmx} % Times
\usepackage[scaled=.90]{helvet}
\usepackage{courier}

%%% User specified packages and stylesheets

%%% Fallback definitions for Docutils-specific commands
% numeric or symbol footnotes with hyperlinks
\providecommand*{\DUfootnotemark}[3]{%
  \raisebox{1em}{\hypertarget{#1}{}}%
  \hyperlink{#2}{\textsuperscript{#3}}%
}
\providecommand{\DUfootnotetext}[4]{%
  \begingroup%
  \renewcommand{\thefootnote}{%
    \protect\raisebox{1em}{\protect\hypertarget{#1}{}}%
    \protect\hyperlink{#2}{#3}}%
  \footnotetext{#4}%
  \endgroup%
}

% hyperlinks:
\ifthenelse{\isundefined{\hypersetup}}{
  \usepackage[colorlinks=true,linkcolor=blue,urlcolor=blue]{hyperref}
  \urlstyle{same} % normal text font (alternatives: tt, rm, sf)
}{}
\hypersetup{
  pdftitle={HOTS AND COLDS Thermometer},
}

%%% Title Data
\title{\phantomsection%
  \textbf{HOTS AND COLDS Thermometer}%
  \label{your-hots-and-colds-thermometer}}
\author{}
\date{}

\pagestyle{plain}
\setlength{\parindent}{0pt}
\setlength{\parskip}{0.3ex}

\newcommand{\HCT}{\textbf{HOTS AND COLDS Thermometer}}


\newcommand{\ButtonEnter}%
           {\includegraphics[height=1.1em]{button-enter.png}}
\newcommand{\ButtonCancel}%
           {\includegraphics[height=1.1em]{button-cancel.png}}
\newcommand{\ButtonUp}%
           {\includegraphics[height=1.1em]{button-up.png}}
\newcommand{\ButtonDown}%
           {\includegraphics[height=1.1em]{button-down.png}}
\newcommand{\ArrowDown}%
           {\includegraphics[height=1.1em]{arrow-down.png}}
\newcommand{\ArrowLeft}%
           {\includegraphics[height=1.1em]{arrow-left.png}}
\newcommand{\ArrowRight}%
           {\includegraphics[height=1.1em]{arrow-right.png}}
\newcommand{\ArrowUp}%
           {\includegraphics[height=1.1em]{arrow-up.png}}


%%% Body
\begin{document}

\pagestyle{empty}

\setlength{\parindent}{0pt}
\setlength{\parskip}{1ex plus 0.5ex minus 0.5ex}

\begin{center}

  \vfill

  \Huge{\HCT}

  \vfill

  \includegraphics[width=\textwidth]{dev-hardware.png}

  {\small FIXME put in the right picture}

  \vfill

\end{center}

\break

Congratulations on your purchase of a new \HCT{}.

You can now display temperature in your home with the user friendly
\textbf{hots and colds} scale.  This scale is centred on a comfortable
temperature for normal people, which is used as a reference for
temperatures both hotter and colder than that.

Temperatures are displayed as:
\begin{itemize}
\item \texttt{NORMAL} for a normal temperature \DUfootnotemark{id1}{id2}{1}
\DUfootnotetext{id2}{id1}{1}{%
Normal is equivalent to 20\textsuperscript{O}C
}
\item A number of hots for temperatures above \texttt{NORMAL} (eg
  \texttt{12 HOTS})
\item A number of colds for temperatures below \texttt{NORMAL} (eg
  \texttt{3 COLDS})
\end{itemize}
%

%\includegraphics[height=147bp,width=413bp]{HC-demo-nobat.png}
\includegraphics[height=0.356\textwidth,width=\textwidth]{HC-demo-nobat.png}

\tableofcontents


%___________________________________________________________________________

\section*{\phantomsection%
  Setting up%
  \addcontentsline{toc}{section}{Setting up}%
  \label{setting-up}%
}

Remove your \HCT{} from its packaging and either stand it on its base in a
convenient location or use the magnet on its rear to attach it to a
suitable surface (eg your fridge.)

Your \HCT{} has been temperature calibrated and its clock has been set at
the factory.


%___________________________________________________________________________

\section*{\phantomsection%
  Precautions%
  \addcontentsline{toc}{section}{Precautions}%
  \label{precautions}%
}

Your \HCT{} is safe for normal use, but there are some precautions you
should take.
%
\begin{itemize}

\item Exposure to extreme temperatures outside the operating range (see
  Specifications) may result in incorrect operation or failure.  At very
  low temperatures (such as in your freezer) the display will appear to
  change very slowly.

\item Do not get the device wet.  If you put it in your freezer, place it
  in a plastic bag first, and only remove it from the plastic bag when the
  bag and device are back at room temperature.

\item Do not leave in direct sunlight.  This will result in incorrect
  temperature readings.

\item Not for use where failure could result in injury or death.

\item Not for medical use.

\item Do not swallow.

\end{itemize}


%___________________________________________________________________________

\section*{\phantomsection%
  The display%
  \addcontentsline{toc}{section}{The display}%
  \label{the-display}%
}

\includegraphics[height=0.356\textwidth,width=\textwidth]{HC-display-parts-nobat.png}

The display of your \HCT{} shows the current time at the top right, and the
current temperature on the main display at the bottom.  The main display is
also used for the menu system, and to display recorded maximum and minimum
temperatures.

While in the menu system, a timeout showing four bars shows at the top left
of the display.  When the timeout bars disappear after 12 seconds without
button presses, the display will exit the menu system.

Four arrow indicators tell you which buttons are active at any time.
\ArrowUp{} means the up botton (\ButtonUp), \ArrowDown{} means the down
button (\ButtonDown), \ArrowLeft{} means the cancel button (\ButtonCancel),
and \ArrowRight{} means the enter button (\ButtonEnter).


%___________________________________________________________________________

\section*{\phantomsection%
  The buttons%
  \addcontentsline{toc}{section}{The buttons}%
  \label{the-buttons}%
}

Your \HCT{} has four buttons.

%\includegraphics[height=65bp,width=413bp]{HC-buttons.png}
\includegraphics[height=0.157\textwidth,width=\textwidth]{HC-buttons.png}

\newcounter{listcnt0}
\begin{list}{\arabic{listcnt0}.}
{
\usecounter{listcnt0}
\setlength{\rightmargin}{\leftmargin}
}

\item \ButtonCancel{} is for the cancel function.

\item \ButtonUp{} is for the up function, and to display maximum
  temperature.

\item \ButtonDown{} is for the down function, and to display minimum
  temperature.

\item \ButtonEnter{} is for the enter function, and for starting the menu.

\end{list}


%___________________________________________________________________________

\section*{\phantomsection%
  Configuring%
  \addcontentsline{toc}{section}{Configuring}%
  \label{configuring}%
}


%___________________________________________________________________________

\subsection*{\phantomsection%
  Setting time%
  \addcontentsline{toc}{subsection}{Setting time}%
  \label{setting-time}%
}
%
\begin{itemize}

\item Press \ButtonEnter{}.

\item The display shows \texttt{SETTIME}.

\item Press \ButtonEnter{} again.

\item The display shows \texttt{Y 2014} (or the current year), and the year
  will flash.

\item Press \ButtonUp{} or \ButtonDown{} to change the year.

\item Press \ButtonEnter{} to go to month, then to day, then to hour, then
  to minute.  Each time element can be changed with \ButtonUp{} and
  \ButtonDown{}.

\item After you press \ButtonEnter{} when setting the minutes, the time is
  set and saved.

\item You can cancel setting the time with \ButtonCancel{}.

\item If you don't press a button for 12 seconds, the display will exit the
  menu system without setting the time.

\end{itemize}


%___________________________________________________________________________

\subsection*{\phantomsection%
  Calibrating temperature%
  \addcontentsline{toc}{subsection}{Calibrating temperature}%
  \label{calibrating-temperature}%
}

Your \HCT{} was calibrated at the factory, so you
should not need to do your own calibration.  But if you wish to change the
calibration:
%
\begin{itemize}

\item Press \ButtonEnter{} to enter the menu.

\item The display shows \texttt{SETTIME}.

\item Press \ButtonDown{}.

\item The display shows \texttt{CALTEMP}.

\item Press \ButtonEnter{}.

\item The display will now show the temperature in the archaic ``degrees
  Celsius'' scale, and an offset.  You can compare the temperature with an
  old style thermometer, and adjust in 0.5\textsuperscript{O}C steps.

\end{itemize}

%\includegraphics[height=147bp,width=413bp]{HC-temp-cal-plus-nobat.png}
\includegraphics[height=0.356\textwidth,width=\textwidth]{HC-temp-cal-plus-nobat.png}

\begin{itemize}

\item When you are happy with the calibration, press \ButtonEnter{}.

\item The calibration is saved permanently.

\item You can cancel the temperature calibration changes by pressing
  \ButtonCancel{}.

\item If you don't press a button for 12 seconds, the display will exit the
  menu system without saving the new temperature calibration.

\end{itemize}


%___________________________________________________________________________

\subsection*{\phantomsection%
  Adjusting time%
  \addcontentsline{toc}{subsection}{Adjusting time}%
  \label{adjusting-time}%
}

Your \HCT{} was adjusted to keep correct time at the factory, and should
not drift more than a couple of minutes per year.  But if you wish to
change its time adjustment:
%
\begin{itemize}

\item Press \ButtonEnter{} to enter the menu.

\item The display says \texttt{SETTIME}.

\item Press \ButtonDown{} twice.

\item The display says \texttt{ADJTIME}.

\item Press \ButtonEnter{}.

\item The display will show the current adjustment in seconds per day.  To
  make the clock run slower, decrease the number (or make it more
  negative.)  To make the clock run faster, increase the number (or make it
  less negative.)

\item You can change the adjustment from -20 seconds per day to +20 seconds
  per day, in steps of 0.25 seconds per day.

\item When you are happy with the adjustment, press \ButtonEnter{}.

\item The time adjustment is saved permanently.

\item You can cancel the time adjustment changes by pressing
  \ButtonCancel{}.

\item If you don't press a button for 12 seconds, the display will exit the
  menu system without saving the new time adjustment.

\end{itemize}


%___________________________________________________________________________

\section*{\phantomsection%
  Displaying the maximum or minimum temperature%
  \addcontentsline{toc}{section}{Displaying the maximum or minimum temperature}%
  \label{displaying-the-maximum-or-minimum-temperature}%
}

When your \HCT{} is displaying the current temperature, you can press
\ButtonUp{} to display the maximum temperature since the last 3pm time.  It
will also display the maximum temperature in the 24 hours to the last 3pm
time, then return to displaying the current temperature.

Press \ButtonDown{} while the display is showing the current temperature,
or while displaying the maximum, to display the minimum temperature since
the last 9am time, followed by the minimum temperature in the 24 hours up
to the last 9am time.

The times and temperatures are shown sequentially, as they don't fit on the
display together.  For example:

\begin{center}

\includegraphics[height=0.136\textwidth,width=0.7\textwidth]{HC-max-MAX_T__.png}

\includegraphics[height=0.136\textwidth,width=0.7\textwidth]{HC-max-SINCE__.png}

\includegraphics[height=0.136\textwidth,width=0.7\textwidth]{HC-max-3_PM___.png}

\includegraphics[height=0.136\textwidth,width=0.7\textwidth]{HC-max-5__HOTS.png}

\end{center}

%___________________________________________________________________________

\section*{\phantomsection%
  Changing the battery%
  \addcontentsline{toc}{section}{Changing the battery}%
  \label{changing-the-battery}%
}

Your \HCT{} requires two 1.5V AAA size batteries.  When the display goes
dim, the batteries should be changed.

To change the batteries, use a pozidrive screwdriver to remove the front
and rear covers.  There is one battery under the front cover and one under
the rear cover.  If the old batteries were not too flat, your \HCT{} will
continue to run for a few minutes while you insert new batteries.

Energizer Lithium batteries should last for at least three years in normal
use.


%___________________________________________________________________________
\break

\section*{\phantomsection%
  Specifications%
  \addcontentsline{toc}{section}{Specifications}%
  \label{specifications}%
}

\setlength{\DUtablewidth}{\linewidth}
\begin{longtable*}[c]{|p{0.450\DUtablewidth}|p{0.450\DUtablewidth}|}
\hline

Minimum temperature & 50 COLDS \\
\hline

Maximum temperature & 50 HOTS \\
\hline

Temperature calibration range &
+/- 4.5 HOTS/COLDS (+/- 4.5\textsuperscript{O}C -{}- Eww) \\
\hline

Time accuracy (with adjustment) &
+/- 0.125 seconds per day (about 45 seconds per year) \\
\hline

Time adjustment range & +/- 20 seconds per day \\
\hline

Battery life & At least 3 years, with lithium AAA batteries \\

\hline

\end{longtable*}


%___________________________________________________________________________

\section*{\phantomsection%
  Warranty%
  \addcontentsline{toc}{section}{Warranty}%
  \label{warranty}%
}

Your \HCT{} has a lifetime warranty.  When it stops working, it is at the
end of its life.

In order to fix identified issues with the device, we may at times offer
software upgrades.


%___________________________________________________________________________
\break

\section*{\phantomsection%
  Calibration Information%
  \addcontentsline{toc}{section}{Calibration Information}%
  \label{calibration-information}%
}
%
\begin{longtable*}[c]{|p{0.450\DUtablewidth}|p{0.450\DUtablewidth}|}
\hline
\texttt{Serial number} & \vskip 2.5em \\
\hline
\texttt{Temperature calibration} & \vskip 2.5em \\
\hline
\texttt{Time adjustment} & \vskip 2.5em \\
\hline
\texttt{Calibrated by} & \vskip 2.5em \\
\hline
\end{longtable*}


\end{document}
